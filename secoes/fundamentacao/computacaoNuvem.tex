Computação em nuvem é um modelo elaborado para disponibilizar acesso conveniente, ubíquo e sob demanda via rede a um conjunto compartilhado de recursos computacionais (redes, servidores, armazenamento, aplicações e serviços) que possa ser rapidamente alocado e disponibilizado com pouco esforço para gerência e mínima interação com provedores de serviço \cite{mell2011nist}.

Em CN, existem três modelos de fornecimento de serviços que podem ser adotados \cite{mell2011nist}, sendo eles:

\begin{itemize}
	\item \textit{Software as a service} \abreviatura{SaaS}{Software as a Service}(SaaS): providência aos consumidores a capacidade de executar programas de provedores na infraestrutura da nuvem. Estas aplicações podem ser acessadas através de interfaces leves, tais como navegadores. O consumidor não despende recursos com a gerência e manutenção da infraestrutura da nuvem, tais como sistema operacional, redes, servidores ou mesmo configurações individuais da aplicação a não ser configurações de execução específicas para usuário.
	
	\item \textit{Platform as a service} \abreviatura{PaaS}{Platform as a Service}(PaaS): providência ao consumidor a capacidade de publicar na nuvem aplicações criadas a partir de linguagens de programação, bibliotecas, serviços e ferramentas suportadas pelo provedor.O cliente não gerencia nem tem controle sobre a infraestrutura da nuvem, incluindo a rede, servidores, sistema operacional e armazenamento. Entretanto, ele tem controle das suas aplicações e possivelmente configurações sobre o ambiente em que é executada a aplicação;

	\item \textit{Infrastructure as a service} \abreviatura{IaaS}{Infrastructure as a Service}(IaaS): providência ao consumidor a possibilidade de alocar processamento, armazenamento, serviços de rede e outros recursos computacionais onde o consumidor pode publicar e executar software arbitrário, o que pode incluir sistemas e aplicações. O consumidor não gerencia nem tem controle sobre a infraestrutura física, mas tem controle sobre sistemas operacionais, armazenamento e de suas aplicações publicadas. Pode existir também casos em que o consumidor tem controle de componentes específicos da rede, tais como firewall.
\end{itemize}

Este trabalho tem como alvo as nuvens que provém o serviço de IaaS, dado que esse tipo de nuvem deve dividir lógicamente seu poder computacional para atender os consumidores. Dessa forma, deve haver a preocupação de segmentar os recursos de forma eficiente, sem a utilização de servidores excedentes ou ainda realocar segmentos, para que seja possível atender uma nova demanda nos servidores ativos.
