A cooperação e a comunicação entre agentes torna-se interessante quando existe a intenção de resolver um problema de forma distribuída, como é o caso da orquestração de um ambiente de computação em nuvem. Existem vários métodos de comunicação entre agentes fundamentalmente diferentes, sendo que \citeonline{intelligent} classificam como o mais simples dos métodos a invocação do procedimento de um agente por outro agente.

Utilizando a invocação de procedimentos, o agente invocador (agente 1) usa parâmetros de invocação explícitos para informar o outro agente (agente 2) de suas intenções. Os valores de retorno do agente 2 representam a resposta da comunicação. Porém, apenas comunicações muito simples podem ser efetuadas invocando procedimentos remotos, sendo que para casos com um mínimo de complexidade os métodos de comunicação entre agentes podem ser diferenciados em sistemas \emph{blackboard} (quadro negro) e sistemas baseados em diálogo (troca de mensagens) \cite{intelligent}.

O conceito de quadro negro surgiu a partir da intenção de lidar com aplicações complexas e fracamente definidas, dando mais flexibilidade a pesquisadores e desenvolvedores, liberando-os de especificações formais excecivas \cite{blackboard}. Um quadro negro proporciona a todos os agentes dentro de um sistema uma área comum de trabalho, a qual eles podem utilizar para trocar informações \cite{intelligent}. Um agente inicia a comunicação escrevendo um item de informação qualquer no quadro negro. Este item fica então disponível para todos os outros agentes do sistema, sendo que qualquer agente pode acessar o quadro negro a qualquer momento para verificar se novas informações surgiram desde sua última verificação \cite{intelligent}. Neste modelo os agentes não precisam saber dos conhecimentos e nem da existência dos outros agentes dentro do sistema, mas devem ser capazes de compreender as informações contidas no quadro negro \cite{blackboard}.

A alternativa ao quadro negro vem na forma de troca de mensagens. A comunicação via troca de mensagens proporciona uma base flexível para a implementação de estratégias complexas de coordenação de agentes, sendo que as mensagens trocadas entre os agentes podem ser usadas para estabelecer comunicações e mecanismos de cooperação usando protocolos pré-definidos \cite{intelligent}. O fato de que diferentes agentes podem ser invocados para realizarem funções diferentes dentro do ambiente, mas em algum ponto de suas respectivas funções, necessitar de alguma informação da qual outro agente pode ter conhecimento \cite{handbook-intelligence} é algo que torna essa abordagem interessante em um ambiente de computação em nuvem.
