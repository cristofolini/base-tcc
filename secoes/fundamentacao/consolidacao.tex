Melhorar a eficiência energética é uma das maiores dificuldades da computação em nuvem \cite{challenges}, sendo estimado que 53\% de todos os gastos de data centers são voltados a energia e refrigeração \cite{micro-slice}. Aderindo a tendências globais de busca de abordagens sustentáveis, surgiu a ideia de se buscar novas formas de diminuir o consumo de energia em ambientes de computação em nuvem, que se tornou uma destacada área de pesquisa voltada a identificar oportunidades e estratégias para a redução do consumo energético, mantendo suporte a cargas dinâmicas e qualidade de serviço.

Uma das estratégias encontradas para uma melhor eficiência energética é a consolidação. Assim como em \citeonline{tcc-gabriel}, no âmbito deste trabalho, consolidação será tratada como a agregação de máquinas virtuais em servidores físicos, de forma a concentrar um número maior de máquinas virtuais em um número menor de servidores físicos. Essa forma de agregação permite a desativação de recursos ociosos, levando a um menor consumo energético.

O benefício de técnicas de consolidação vem na possibilidade de aumentar a eficiência de um ambiente de CN através do desligamento de recursos ociosos. Em um cenário ótimo, todos os servidores que estão ligados, tendem a carga máxima, fazendo com que a eficiência deste ambiente também esteja tendendo ao máximo possível. Para que seja possível a consolidação, deve ser possível migrar VMs em tempo de execução, fato que só ocorre entre servidores com hardware compatível e mesmo virtualizador. Este e outros fatores, como a necessidade de atender a cargas estocásticas, faz com que a consolidação de ambientes em nuvem não seja um problema trivial de se resolver.\par

Para que a consolidação seja possível, deve haver alguma forma de comunicação entre servidores, para que estes acordem em realizar um rebalanceamento de carga. Além disso, a comunicação e migração deve ocorrer entre máquinas que se encaixam no cenário em que há a necessidade de uma migração em tempo de execução. Neste tipo de ambiente, além de estarem presentes conceitos de negociação e atos de fala, há informação fragmentada inserida nos elementos do ambiente. Tais pontos são discutidos e estão presentes no conjunto de características de problemas onde o uso de sistemas multiagente é recomendado, apontadas por Wooldridge em seu livro \cite{wooldridge2009introduction}.
