Dentre as ferramentas para modelagem de agentes disponíveis, a \emph{Prometheus Design Tool} \cite{pdt} chama atenção por ter sido elaborada em conjunto com os autores da metodologia \emph{Prometheus}, justamente para complementar a metodologia.

A \emph{Prometheus Design Tool} é baseada em características específicas de agentes como metas, planos, percepções, ações e protocolos e é estruturada em torno das três fases da metodologia \emph{Prometheus} descritas anteriormente \cite{pdt}. Além do suporte ao desenvolvimento gráfico de diagramas de \emph{design} para o sistema multiagente, \citeonline{pdt} citam diferentes atributos da ferramenta, sendo alguns dos mais notáveis destes:

\begin{itemize}
  \item Verificação de Consistência: a ferramenta mantém restrições baseados em um meta-modelo, providenciando suporte à prevenção de erros simples como a geração de uma entidade indesejada causada por um erro de digitação;
  \item Propagação: sempre que possível, informações são propagadas de uma parte do modelo para outra. Por exemplo, se um conjunto de metas é associado a um papel, e esse papel é associado a um agente, então as metas serão automaticamente associadas ao agente;
  \item Testes Automatizados: a ferramenta suporta a geração e execução automatizadas de testes de unidade para eventos, crenças e planos, baseados no modelo.
\end{itemize}
