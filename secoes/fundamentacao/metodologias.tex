A partir dos conceitos de modelagem de agentes apresentados, foram elaboradas diversas metodologias para a concepção e o desenvolvimento destes agentes. Em função da sua utilidade em relação ao desenvolvimento de agentes \emph{BDI}, sem deixar de ser útil para qualquer outro tipo de sistema multiagente, a metodologia utilizada no decorrer deste trabalho é a chamada \emph{Prometheus} \cite{prometheus}.

A metodologia \emph{Prometheus} consiste de três fases, descritas por \citeonline{prometheus} da seguinte forma:
\begin{itemize}
  \item Especificação de Sistema: é focada em identificar as funcionalidades básicas do sistema como um todo, tais como suas entradas (percepções), saídas (ações), e quaisquer fontes importantes de dados compartilhados;
  \item Modelagem Arquitetural: utiliza as saídas da fase anterior para determinar quais agentes o sistema conterá e como eles irão interagir;
  \item Modelagem Detalhada: é onde são especificados os componentes internos de cada agente e como tal agente realizará suas tarefas dentro do sistema.
\end{itemize}

\begin{figure}[!htb]
	\centering
	\caption{Diagrama explicitando as fases da metodologia \emph{Prometheus}, obtido de \cite{promeheus}.}\label{fig:promehteus}
	\includegraphics[width=1\textwidth]{figuras/promehteus.gif}
\end{figure}

Vale notar que estas fases descrevem um processo iterativo de engenharia de software, e não um modelo de desenvolvimento do tipo cascata, sendo que estas fases não devem ser necessariamente executadas em alguma ordem em particular. \citeonline{prometheus} sugerem repetir o processo inteiro mais de uma vez, com um foco diferente a cada repetição, sendo que a primeira iteração pode consistir inteiramente de atividades associadas a fase de especificação de sistema, iterações subsequentes envolverão uma mistura de atividades de fases diferentes, eventualmente com uma presença maior de atividades das fases seguintes. 
