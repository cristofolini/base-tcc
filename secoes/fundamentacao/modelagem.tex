Sistemas baseados em agentes podem ser modelados de forma similar a sistemas orientados a objeto, sendo que seus agentes tomam forma de objeto e passam a ser constituídos por atributos e métodos podendo se comunicar invocando métodos de outros agentes ou trocando mensagens, e podendo utilizar conceitos clássicos de orientação a objeto, como herança, encapsulamento e agregação de dados \cite{intelligent}. Em consequência disso, métodos relacionados ao desenvolvimento de aplicativos orientados a objeto formam a base de boa parte dos métodos de desenvolvimento de sistemas baseados em agentes. De forma similar a técnica para modelagem orientada a objeto proposta por \citeonline{OMT} que consiste de modelo básico, modelo estático e modelo dinâmico, \citeonline{agent-oriented} divide a análise orientada a agentes em três modelos, o modelo de agente, o modelo organizacional e o modelo cooperativo, descritos a seguir:

\begin{itemize}
  \item O modelo de agente contém descrições e estruturas internas dos agentes, descritas em termos de noções mentais como metas planos e crenças ou quaisquer estruturas que sejam apropriadas a arquitetura específica doas agentes sendo desenvolvidos. Esse modelo se assemelha ao modelo básico de métodos orientados a objeto;
  \item O modelo organizacional especifica os relacionamentos entre agentes e seus tipos. Estes são em parte relações de herança, e também relacionamentos entre agentes baseados em seus respectivos papéis em organizações. Essas organizações podem ser meios para estruturar sistemas complexos em subsistemas (assim como é feito em certas técnicas de orientação a objeto) ou podem ser usadas para modelar organizações reais. Esse modelo é semelhante ao modelo estático, mas como papéis podem mudar com o passar do tempo, ele não é um modelo genuinamente estático;
  \item O modelo cooperativo descreve a interação, ou mais especificamente, a cooperação entre os agentes. Este modelo contém apenas os \hyphenation{relacio-namentos} relacionamentos entre objetos. O processo que ocorre dentro dos objetos é representado pelo modelo de agente.
\end{itemize}
