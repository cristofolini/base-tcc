Sistemas multiagentes � considerada uma �rea de estudo dentro da Intelig�ncia Artificial Distribu�da (IAD)\abreviatura{IAD}{Intelig�ncia Artificial Distribu�da}. SMA possuem a capacidade de lidar com problemas em ambientes distribu�dos e abertos, tal como os ambientes em larga escala encontrados que usam a internet como meio \cite{wooldridge2009introduction}.\par

A abordagem de SMA � descrita na forma de m�ltiplos elementos computacionais (agentes) que trocam conhecimento entre si na forma de coopera��o, coordena��o, negocia��o e similares, possibilitando a fragmenta��o de problemas complexos em sub-problemas menores e objetivos. Estes, por sua vez, podem ser abordados de diferentes formas, por diferentes agentes especialistas \cite{a-ricardo-intro}.\par

O emprego de SMA tem apresentado sucesso em �reas que trabalham com ambientes din�micos e descentralizados, onde a tomada de decis�o n�o depende apenas de um �nico ponto de vista \cite{kelash2007takes}.
