% Identificadores do trabalho
% Usados para preencher os elementos pré-textuais

\instituicao[a]{Universidade Federal de Santa Catarina} % Opcional
\departamento[o]{INE}
\curso[o]{curso de Ciências da Computação}
\documento[o]{TCC} % [o] para dissertação [a] para tese
\titulo{Proposta de uma plataforma de sistema multiagente para suportar ações visando otimização energética em ambientes de computação em nuvem}
\autor{Lucas Berri Cristofolini}
\grau{Bacharel}
\local{Florianópolis} % Opcional (Florianópolis é o padrão)
\data{23}{agosto}{2015}
\orientador[Orientador\\Universidade Federal de Santa Catarina]{Prof. Dr. Ricardo Azambuja Silveira}
\coorientador[Coorientador\\Universidade Federal de Santa Catarina]{Rafael Weingärtner}
\coordenador[Coordenador\\Universidade Federal de Santa Catarina]{Prof. Dr. Renato Cislaghi}

\numerodemembrosnabanca{4} % Isso decide se haverá uma folha adicional
\orientadornabanca{nao} % Se faz parte da banca definir como sim
\coorientadornabanca{sim} % Se faz parte da banca definir como sim
\bancaMembroA{Primeiro membro\\Universidade ...} %Nome do presidente da banca
\bancaMembroB{Segundo membro\\Universidade ...}      % Nome do membro da Banca
\bancaMembroC{Terceiro membro\\Universidade ...}     % Nome do membro da Banca
%\bancaMembroD{Quarto membro\\Universidade ...}       % Nome do membro da Banca
%\bancaMembroE{Quinto membro\\Universidade ...}       % Nome do membro da Banca
%\bancaMembroF{Sexto membro\\Universidade ...}        % Nome do membro da Banca
%\bancaMembroG{Sétimo membro\\Universidade ...}       % Nome do membro da Banca

\dedicatoria{Este trabalho é dedicado aos meus colegas de classe e aos meus queridos pais.}

\agradecimento{Inserir os agradecimentos aos colaboradores à execução do trabalho.}

\epigrafe{Texto da Epígrafe. Citação relativa ao tema do trabalho. É opcional. A epígrafe pode também aparecer na abertura de cada seção ou capítulo.}
{(Autor da epígrafe, ano)}