Ambientes de computação em nuvem mantém recursos ociosos ativos visando manter a disponibilidade e a qualidade do serviço em momentos de aumento súbito de utilização de recursos \cite{tcc-gabriel}. Isso faz com que o consumo energético destes ambientes seja elevado, causando um impacto considerável nas despesas envolvidas em manutenção bem como no meio-ambiente, considerando que apenas 21\% de toda energia produzida mundialmente procede de fontes renováveis \cite{ieo2013}. \par

A energia usada por um servidor operante não é linearmente proporcional a carga com que ele está trabalhando, como observado em \citeonline{awada2014energy}. Pontos levantados na pesquisa realizada por Awada, Li e Shen servem de motivação para o estudo de técnicas para gerência de ambientes na nuvem, entre eles, alguns merecem destaque \cite{tcc-gabriel}:

\begin{itemize}  
	\item Os servidores, mesmo com 20\% de carga, tendem a consumir 80\% da energia necessária quando exigidos ao máximo;
	\item Servidores, avaliados em ambientes reais, no geral são expostos a cargas de uso entre 10\% e 50\%;
	\item O custo para refrigeracão destes ambientes, reflete em 30\% do custo energético total do centro de processamento.
\end{itemize}

Estes dados mostram que é necessário uma melhor gerência dos recursos disponíveis. Um /textit{datacenter} que possui servidores sub-utilizadas, torna-se ineficiênte, tanto energeticamente quanto pelo poder computacional ocioso. Até então, nenhuma das ferramentas de orquestração de CN mais conhecidas e disponíveis no mercado avaliadas por \citeonline{tcc-gabriel} possuem funcionalidades voltadas à gerência do consumo energético nos ambientes orquestrados o que faz com que empresas de pequeno e médio porte, que representam mais do que 90\% do consumo energético utilizado para alimentação e refrigeração de data centers americanos \cite{whitney2014data}, tenham dificuldades para gerenciar o consumo energético dos seus data centers.\par

Seguindo os passos de agentes informais de consolidação já propostos e testados por Uchechukwu, Li e Shen, que mostram resultados de diminuição de máquinas ativas de 81\% para 44\% \cite{uchechukwu2012improving}, pode-se pensar em uma implementação formal, com parâmetros variáveis de QoS e distribuída, usando técnicas de sistemas multiagentes.\par

Por mais que o uso de agentes na nuvem tenha sido, até agora, limitado a SMAs que usam recursos computacionais da nuvem e nuvens que utilizam SMA para prover serviços inteligentes \cite{a-prieta}, Wooldridge pontua que o uso de uma abordagem multiagentes é recomendada quando o problema possui um ambiente aberto, dinâmico e complexo, ou também quando existem dados, controle e expertise distribuidos no ambiente. \par

Tendo em vista a adequação de um sistema multiagente a um meio heterogêneo como a orquestração de um ambiente de CN, surgiu o interesse em expandir o \emph{framework} proposto por \citeonline{tcc-gabriel} com a adoção de uma plataforma multiagente que viria a substituir o agente implementado. Assim, flexibiliza-se a adaptação dos agentes para a execução de diferentes ações nas diferentes partes que compõem a ferramenta de orquestração, visando uma maior especificidade e paralelização das tarefas a serem adotadas. \par

A adoção de uma plataforma multiagente permitirá a implementação de diferentes agentes que atuarão nas diferentes partes heterogêneas que compõem um ambiente de CN. Além de serem adaptados a áreas específicas do ambiente, esses agentes terão a capacidade de se comunicar, permitindo que agentes voltados para uma parte do ambiente possam reagir adequadamente a ocorrências de eventos em uma parte diferente do ambiente. Diferentes agentes poderão ser inicializados e removidos de acordo com a demanda do ambiente no qual forem implementados. \par
