%%%%%%%%%%%%%%%%%%%%%%%%%%%%%%%%%%%%%%%%%%%%%%%%%%%%%%%%%%%%%%%%%%%%%%%
% Universidade Federal de Santa Catarina             
% Biblioteca Universit�ria                     
%----------------------------------------------------------------------
% Exemplo de utiliza��o da documentclass ufscThesis
%----------------------------------------------------------------------                                                           
% (c)2013 Roberto Simoni (roberto.emc@gmail.com)
%         Carlos R Rocha (cticarlo@gmail.com)
%         Rafael M Casali (rafaelmcasali@yahoo.com.br)
%%%%%%%%%%%%%%%%%%%%%%%%%%%%%%%%%%%%%%%%%%%%%%%%%%%%%%%%%%%%%%%%%%%%%%%
\documentclass{ufscThesis} % Definicao do documentclass ufscThesis	

%----------------------------------------------------------------------
% Pacotes usados especificamente neste documento
\usepackage{graphicx} % Possibilita o uso de figuras e gr�ficos
\usepackage{color}    % Possibilita o uso de cores no documento
\usepackage{listings}
%----------------------------------------------------------------------
% Comandos criados pelo usu�rio
\newcommand{\afazer}[1]{{\color{red}{#1}}} % Para destacar uma parte a ser trabalhada
\newcommand{\ABNTbibliographyname}{REFER�NCIAS}

%----------------------------------------------------------------------

\instituicao[a]{Universidade Federal de Santa Catarina} % Opcional
\departamento[o]{INE}
\curso[o]{curso de Ci�ncias da Computa��o}
\documento[o]{TCC} % [o] para disserta��o [a] para tese
\titulo{Proposta de uma plataforma de sistema multiagente para suportar a��es visando otimiza��o energ�tica em ambientes de computa��o em nuvem}
\autor{Lucas Berri Cristofolini}
\grau{Bacharel}
\local{Florian�polis} % Opcional (Florian�polis � o padr�o)
\data{23}{agosto}{2015}
\orientador[Orientador\\Universidade Federal de Santa Catarina]{Prof. Dr. Ricardo Azambuja Silveira}
\coorientador[Coorientador\\Universidade Federal de Santa Catarina]{Rafael Weing�rtner}
\coordenador[Coordenador\\Universidade Federal de Santa Catarina]{Prof. Dr. Renato Cislaghi}

\numerodemembrosnabanca{4} % Isso decide se haver� uma folha adicional
\orientadornabanca{nao} % Se faz parte da banca definir como sim
\coorientadornabanca{sim} % Se faz parte da banca definir como sim
\bancaMembroA{Primeiro membro\\Universidade ...} %Nome do presidente da banca
\bancaMembroB{Segundo membro\\Universidade ...}      % Nome do membro da Banca
\bancaMembroC{Terceiro membro\\Universidade ...}     % Nome do membro da Banca
%\bancaMembroD{Quarto membro\\Universidade ...}       % Nome do membro da Banca
%\bancaMembroE{Quinto membro\\Universidade ...}       % Nome do membro da Banca
%\bancaMembroF{Sexto membro\\Universidade ...}        % Nome do membro da Banca
%\bancaMembroG{S�timo membro\\Universidade ...}       % Nome do membro da Banca

\dedicatoria{Este trabalho � dedicado aos meus colegas de classe e aos meus queridos pais.}

\agradecimento{Inserir os agradecimentos aos colaboradores � execu��o do trabalho.}

\epigrafe{Texto da Ep�grafe. Cita��o relativa ao tema do trabalho. � opcional. A ep�grafe pode tamb�m aparecer na abertura de cada se��o ou cap�tulo.}
{(Autor da ep�grafe, ano)}

\textoResumo {Dada a complexidade, heterogeneidade e dinamismo crescentes presentes nos ambientes de computa��o em nuvem, otimizar a utiliza��o de recursos nestes ambientes torna-se uma tarefa desafiadora. Motivado pela semelhan�a entre entre os paradigmas de computa��o em nuvem e sistemas multiagentes, este trabalho se prop�e a adaptar uma ferramenta de orquestra��o existente, de modo a utilizar uma plataforma de sistemas multiagentes para possibilitar que agentes inteligentes realizem a an�lise, o planejamento e a ger�ncia em ambientes de computa��o em nuvem de forma independente e aut�noma.}
\palavrasChave {Computa��o em Nuvem. Int�ligencia Artificial.  Sistemas multiagentes. Orquestra��o de computa��o em nuvem}

\textAbstract {Considering the complexity, heterogeneity and growing dynamism presented on cloud computing environments, performing an optimization in such environments turns into a challenging task. This work is driven by the similarities between cloud computing and multiagent systems paradigms and aims to propose the addition of a multiagent systems platform into an existing orchestration tool, thereby enabling autonomous agents to handle the analysis, planning and management of cloud computing environments.}
\keywords {Cloud Computing. Artificial Intelligence. Multiagent Systems}


%----------------------------------------------------------------------
% In�cio do documento                                
\begin{document}
%--------------------------------------------------------
% Elementos pr�-textuais
%\capa  
\folhaderosto % Se nao quiser imprimir a ficha, � s� n�o usar o par�metro
%\folhaaprovacao
%\paginadedicatoria
%\paginaagradecimento
%\paginaepigrafe
\paginaresumo
\paginaabstract
%\pretextuais % Substitui todos os elementos pre-textuais acima
\listadefiguras % as listas dependem da necessidade do usu�rio
%\listadetabelas 
\listadeabreviaturas
%\listadesimbolos
\sumario 
%--------------------------------------------------------
% Elementos textuais

\chapter{Introdu��o}
Computa��o em nuvem (CN)\abreviatura{CN}{Computa��o em nuvem} � um paradigma tecnol�gico, que vem chamando a aten��o dos provedores de servi�os, por propor uma mudan�a na forma de disponibilizar seus produtos. A grande aceita��o aos servi�os de CN vista recentemente se deve, entre outros motivos, � n�o necessidade de um grande investimento inicial, sendo que os clientes que buscam esse servi�o pagam apenas pelo que utilizam \cite{payg}. A ado��o do paradigma � uma tend�ncia observada em empresas de pequeno � grande porte, sendo as principais raz�es, melhorar a disponibilidade dos seus servi�os al�m de obter uma melhora na infra-estrutura interna \cite{instance1290}.\par

 Os benef�cios da CN est�o altamente ligados com a qualidade de servi�o (QoS) \abreviatura{QoS}{Qualidade de Servi�o}percebida pelos usu�rios. A necessidade de entregar QoS aos consumidores faz com que recursos sejam mantidos ociosos a espera de cargas estoc�sticas, consumindo mais energia e influenciando diretamente nos custos de manuten��o de um ambiente na nuvem \cite{forecasting}.\par

 Uma alternativa para reduzir este consumo desnecess�rio em um ambiente de CN consiste em aglomerar as m�quinas virtuais ativas no menor n�mero de servidores f�sicos, possibilitando desligar os recursos ociosos e lig�-los quando a demanda por recursos tornar-se necess�ria \cite{consolidation}. Entretanto, dada a falta de suporte a medidas de gerenciamento energ�tico nas ferramentas de orquestra��o de ambientes de CN dispon�veis atualmente, \citeonline{gabriel} propos a ado��o de um \emph{framework} para a consolida��o de recursos em ambientes de CN visando uma melhor efici�ncia energ�tica do ambiente  orquestrado. A proposta consiste na implementa��o de um gerente de consolida��o baseado em heur�sticas pr�-definidas para agrupar m�quinas virtuais no menor n�mero de servidores f�sicos poss�veis para poder desativar servidores sem carga.\par 

Problemas como este, onde n�o se conhece uma f�rmula para obter a melhor solu��o e onde a informa��o est� distribu�da pelo ambiente podem ser tratados atrav�s de uma abordadem de sistemas multiagentes (SMA)\abreviatura{SMA}{Sistemas Multiagente}. Dessa forma � poss�vel trabalhar de forma h�brida em cima do problema \cite{a-ricardo-intro}, trazendo o conhecimento de um especialista para os agentes do ambiente, que por sua vez atuar�o para alcan�ar seus objetivos. Adicionado a isto, a natureza heterog�nea, din�mica e cont�nua de SMA, semelhante a de CN, faz com que pare�a natural e condizente uma converg�ncia entre as t�cnologias. 

\section{Motiva��o e Justificativa}
Ambientes de computa��o em nuvem mant�m recursos ociosos ativos visando manter a disponibilidade e a qualidade do servi�o em momentos de aumento s�bito de utiliza��o de recursos \cite{gabriel}. Isso faz com que o consumo energ�tico destes ambientes seja elevado, causando um impacto consider�vel nas despesas envolvidas em manuten��o bem como no meio ambiente, considerando que apenas 21\% de toda energia produzida mundialmente procede de fontes renov�veis \cite{ieo2013}. \par

A energia usada por um servidor ligado n�o � linearmente proporcional a carga com que ele est� trabalhando, como observado em \citeonline{awada2014energy}. Pontos levantados na pesquisa realizada por Awada, Li e Shen servem de motiva��o para o estudo de t�cnicas para ger�ncia de ambientes na nuvem, entre eles, alguns merecem destaque \cite{gabriel}:

\begin{itemize}  
	\item Os servidores, mesmo com 20\% de carga, tendem a consumir 80\% da energia necess�ria quando exigidos ao m�ximo;
	\item Servidores, avaliados em ambientes reais, no geral s�o expostos a cargas de uso entre 10\% e 50\%;
	\item O custo para refrigerac�o destes ambientes, reflete em 30\% do custo energ�tico total do centro de processamento.
\end{itemize}

Esses dados mostram que � necess�rio uma melhor ger�ncia dos recursos dispon�veis. Um centro de processamento de dados que possui servidores sub-utilizadas, torna-se ineficiente, tanto energeticamente quanto pelo poder computacional ocioso. At� ent�o, nenhuma das ferramentas de orquestra��o de CN mais conhecidas e dispon�veis no mercado avaliadas por \citeonline{gabriel} possuem funcionalidades voltadas � ger�ncia do consumo energ�tico nos ambientes orquestrados. Tal cen�rio faz com que empresas de pequeno e m�dio porte, representando mais do que 90\% do consumo energ�tico total utilizado para alimenta��o e refrigera��o de \emph{datacenters} americanos, tenham dificuldades para gerenciar o consumo energ�tico dos seus centros de processamento de dados \cite{whitney2014data}.\par

Seguindo os passos de agentes informais de consolida��o j� propostos e testados por Uchechukwu, Li e Shen, que mostram resultados de diminui��o de m�quinas ativas de 81\% para 44\% \cite{uchechukwu2012improving}, pode-se pensar em uma implementa��o formal, com par�metros vari�veis de QoS, usando t�cnicas de sistemas multiagentes.\par

Por mais que o uso de agentes na nuvem tenha sido, at� agora, limitado a SMAs que usam recursos computacionais da nuvem e nuvens que utilizam SMA para prover servi�os inteligentes \cite{mascloud}. \citeonline{wooldridge2009introduction} pontua que o uso de uma abordagem multiagentes � recomendada quando o problema possui um ambiente aberto, din�mico e complexo, ou tamb�m quando existem dados, controle e expertise distribuidos no ambiente. \par

Tendo em vista a adequa��o de um sistema multiagente a um meio heterog�neo como a orquestra��o de um ambiente de CN, surgiu o interesse em expandir o \emph{framework} proposto por \citeonline{gabriel} com a ado��o de uma plataforma multiagente que viria a substituir o agente implementado. Assim, flexibiliza-se a adapta��o dos agentes para a execu��o de diferentes a��es nas diferentes partes que comp�em a ferramenta de orquestra��o, visando uma maior especificidade e paraleliza��o das tarefas a serem adotadas. \par

A ado��o de uma plataforma multiagente permitir� a implementa��o de diferentes agentes que atuar�o em partes heterog�neas distintas que comp�em um ambiente de CN. Al�m de serem adaptados a �reas espec�ficas do ambiente, esses agentes ter�o a capacidade de se comunicar, permitindo que sejam voltados para uma parte do ambiente onde possam reagir adequadamente a ocorr�ncias de eventos. Diferentes agentes poder�o ser inicializados e removidos de acordo com a demanda do ambiente no qual forem implementados. \par


\section{Objetivos}
\subsection{Objetivo Geral}


\subsection{Objetivos Espec�ficos}


\section{Organiza��o de texto}
Este trabalho ser� dividido em cap�tulos, sendo eles:

\begin{enumerate}
	\item Introdu�ao: apresenta o trabalho, assim como sua motiva��o, objetivos e organiza��o; 
	\item Fundamenta��o te�rica: detalha conceitos utilizados no decorrer deste estudo, assim como os restringe ao escopo do trabalho;
	\item Trabalhos relacionados: apresenta pesquisas recentes, relacionados � proposta apresentada;
	\item Proposta: descreve a plataforma a ser adicionada em uma ferramenta de orquestra��o, apresentando a an�lise da problem�tica, solu��es propostas e algoritmos;
	\item Resultados: apresenta os resultados obtidos a partir da execu��o do trabalho;
	\item Conclus�es: finaliza o trabalho, descrevendo as contribui��es, considera��es e trabalhos futuros.
\end{enumerate}


\chapter{Fundamenta��o}
Neste cap�tulo ser�o introduzidos conceitos abordados no decorrer do trabalho.

\section{Computa��o em Nuvem}
Computa��o em nuvem � um modelo elaborado para disponibilizar acesso conveniente, ub�quo e sob demanda via rede a um conjunto compartilhado de recursos computacionais (redes, servidores, armazenamento, aplica��es e servi�os) que possa ser rapidamente alocado e disponibilizado com pouco esfor�o para ger�ncia e m�nima intera��o com provedores de servi�o \cite{nist}.

Em CN, existem tr�s modelos de fornecimento de servi�os que podem ser adotados \cite{nist}, sendo eles:

\begin{itemize}
	\item \textit{Software as a service} \abreviatura{SaaS}{\textit{Software as a Service}}(SaaS): � fornecido aos consumidores a capacidade de executar programas de provedores na infraestrutura da nuvem. Estas aplica��es podem ser acessadas atrav�s de interfaces leves, tais como navegadores. O consumidor n�o despende recursos com a ger�ncia e manuten��o da infraestrutura da nuvem, tais como sistema operacional, redes, servidores ou mesmo configura��es individuais da aplica��o a n�o ser configura��es de execu��o espec�ficas para usu�rio;
	
	\item \textit{Platform as a service} \abreviatura{PaaS}{\textit{Platform as a Service}}(PaaS): o consumidor do recurso em numve possui a capacidade de publicar na nuvem aplica��es criadas a partir de linguagens de programa��o, bibliotecas, servi�os e ferramentas suportadas pelo provedor. O cliente n�o gerencia nem tem controle sobre a infraestrutura da nuvem, incluindo a rede, servidores, sistema operacional e armazenamento. Entretanto, ele tem controle das suas aplica��es e possivelmente configura��es sobre o ambiente em que � executada a aplica��o;

	\item \textit{Infrastructure as a service} \abreviatura{IaaS}{\textit{Infrastructure as a Service}}(IaaS): o cliente t�m a possibilidade de alocar processamento, armazenamento, servi�os de rede e outros recursos computacionais onde � poss�vel publicar e executar software arbitr�rio, o que pode incluir sistemas operacionais. O consumidor n�o tem a liberdade de gerenciar a infraestrutura f�sica, com exe��o de manter controle sobre armazenamento e suas aplica��es publicadas. Pode existir tamb�m casos em que o consumidor tem controle de componentes espec�ficos da rede, tais como firewall.
\end{itemize}

Este trabalho tem como alvo as nuvens que prov� o servi�o de IaaS. Dessa forma, dado que esse tipo de nuvem deve dividir l�gicamente sua carga de trabalho a fim de atender seus consumidores, deve existir a preocupa��o em segmentar e realocar os recursos de forma eficiente, reduzindo a quantidade de servidores excedentes.


\subsection{Virtualiza��o}
Segundo \citeonline{tholeti2011hypervisors}, virtualiza��o � definida como a cria��o de substitutos, usando divis�es l�gicas, para os verdadeiros recursos f�sicos. Estes possuem a mesma fun��o e interfaces externas, mas diferem em tamanho, performance e custo. Os usu�rios destes recursos virtuais t�picamente n�o percebem a substitui��o, uma vez que sistemas virtuais devem obter performance semelhante a sua contraparte f�sica, em rela��o as aplica��es dentro do servidor. Usando virtualiza��o � poss�vel fazer com que um recurso f�sico seja visto como m�ltiplos recursos virtuais. A figura \ref{fig:virtualizacao} demonstra a virtualiza��o de m�ltiplos sistemas virtuais independentes, que usam recursos virtuais, sobre um �nico sistema f�sico.

\begin{figure}[!htb]
 	\centering
 	\caption{Representa��o de um ambiente virtualizado, obtido de \cite{tholeti2011hypervisors}.}\label{fig:virtualizacao}
 	\includegraphics[width=1\textwidth]{figuras/virtualizacao.png}
\end{figure}


Os virtualizadores s�o componentes de software ou firmware, capazes de virtualizar recursos f�sicos. Existem tr�s tipos de virtualiza��o diferentes, suas defini��es s�o \cite{tholeti2011hypervisors}:

\begin{itemize}
	\item Tipo 1: A virtualiza��o de tipo 1 � caracterizada quando o virtualizador executa diretamente sobre o hardware do sistema. Este m�todo � mais eficiente, tendo o melhor desempenho entre os m�todos de virtualiza��o;
	
	\item Tipo 2: A virtualiza��o de tipo 2 � caracterizada quando o virtualizador executa sobre um sistema operacional hospedeiro, que prov� servi�os de virtualiza��o, tais como suporte � entrada/sa�da de dados e gerenciamento de mem�ria.
	
	\item Virtualiza��o por uso de containers: Este modo de virtualiza��o � caracterizado pela abstin�ncia do virtualizador. A virtualiza��o ocorre a n�vel de sistema operacional\abreviatura{SO}{Sistema Operacional} (SO). Existe um SO raiz, que permite a virtualiza��o de inst�ncias suas. Este modo de virtualiza��o resulta num uso eficiente de mem�ria, sendo esta sua vantagem diante dos outros m�todos de virtualiza��o. Todavia, como todas inst�ncias virtuais s�o imagens do SO ra�z, todas as inst�ncias ser�o da mesma vers�o que o SO ra�z.
	
\end{itemize}

A virtualiza��o � um processo utilizado nos ambientes na nuvem para dividir l�gicamente os recursos f�sicos de um servidor, normalmente sendo usados virtualizadores como componente respons�vel pela cria��o, desaloca��o e gerenciamento das \textit{virtual machines} \abreviatura{VM}{Virtual Machine}(VMs) ou, em portugu�s, m�quinas virtuais. O uso de virtualizadores facilita o provisionamento de diferentes sistemas operacionais sem que eles estejam instalados diretamente nos servidores f�sicos. Outra vantagem da virtualiza��o est� na facilita��o do processo de migra��o de VMs em tempo de execu��o para outros servidores fisicos, facilitando o processo de manuten��o.

\subsection{Efici�ncia energ�tica e consolida��o}
Melhorar a efici�ncia energ�tica � uma das maiores dificuldades da computa��o em nuvem \cite{challenges}, sendo estimado que 53\% de todos os gastos de centro de processamento de dados s�o voltados a energia e refrigera��o \cite{micro-slice}. Aderindo a tend�ncias globais de busca de abordagens sustent�veis, surgiu a ideia de se buscar novas formas de diminuir o consumo de energia em ambientes de computa��o em nuvem, mantendo suporte a cargas din�micas e qualidade de servi�o.

Uma das estrat�gias encontradas para uma melhor efici�ncia energ�tica � a consolida��o. Assim como em \citeonline{gabriel}, no �mbito deste trabalho, consolida��o ser� tratada como a agrega��o de m�quinas virtuais em servidores f�sicos, de forma a concentrar um n�mero maior de m�quinas virtuais em um n�mero menor de servidores f�sicos. Essa forma de agrega��o permite a desativa��o de recursos ociosos, levando a um menor consumo energ�tico.

O benef�cio de t�cnicas de consolida��o vem na possibilidade de aumentar a efici�ncia de um ambiente de CN atrav�s do desligamento de recursos ociosos. Em um cen�rio �timo, todos os servidores que est�o ligados, tendem a carga m�xima, fazendo com que a efici�ncia deste ambiente tamb�m esteja tendendo ao m�ximo poss�vel. Para que seja poss�vel a consolida��o, deve ser poss�vel migrar VMs em tempo de execu��o, fato que s� ocorre entre servidores com hardware compat�vel e mesmo virtualizador. Este e outros fatores, como a necessidade de atender a cargas estoc�sticas, faz com que a consolida��o de ambientes em nuvem n�o seja um problema trivial de se resolver.\par

Para que a consolida��o seja poss�vel, deve haver alguma forma de comunica��o entre servidores, para que estes acordem em realizar um rebalanceamento de carga. Al�m disso, a comunica��o e migra��o deve ocorrer entre m�quinas que se encaixam no cen�rio em que h� a necessidade de uma migra��o em tempo de execu��o. Neste tipo de ambiente, al�m de estarem presentes conceitos de negocia��o e atos de fala, h� informa��o fragmentada inserida nos elementos do ambiente. Tais pontos s�o discutidos e est�o presentes no conjunto de caracter�sticas de problemas onde o uso de sistemas multiagente � recomendado, apontadas por Wooldridge em seu livro \cite{wooldridge2009introduction}.


\subsection{Orquestra��o}
Dadas a complexidade e heterogeneidade de ambientes de computa��o em nuvem, que possuem in�meros componentes e vari�veis din�micas interligadas que necessitam ser orquestradas, a ger�ncia destes ambientes torna-se humanamente imposs�vel \cite{forecasting}. Tendo isso em vista, ferramentas foram criadas com o objetivo de orquestrar a infraestrutura do ambiente, realizando a liga��o dos diferentes componentes como servidores de armazenamento, virtualiza��o e dispositivos de redes para prover a abstra��o de uma nuvem computacional.

\citeonline{tcc-gabriel} cita cinco componentes que comp�e as ferramentas dispon�veis para orquestra��o:
\begin{enumerate} 
 \item Um gerente de identidades que realiza a ger�ncia das credenciais dos usu�rios (autentica��o e autoriza��o), possibilitando maior seguran�a na organiza��o da infraestrutura;
 \item Um gerente de aloca��o que aloca m�quinas virtuais quando s�o instanciadas, ligadas ou migradas;
 \item Um \emph{hypervisor} que � o componente que interage com diferentes virtualizadores, sendo respons�vel por enviar comandos e gerenciar cada cluster;
 \item Um componente de armazenamento que realiza a distribui��o e o compartilhamento dos HDs de m�quinas virtuais entre os servidores;
 \item Um elemento encarregado de gerenciar a configura��o da rede.
\end{enumerate}

A Figura \ref{fig:orquestracao}, retirada desse estudo, mostra de forma visual a arquitetura de uma ferramenta de orquestra��o.

 \begin{figure}[!htb]
 	\centering
 	\caption{Representa��o de uma ferramenta de orquestra��o \cite{tcc-gabriel}.}\label{fig:orquestracao}
 	\includegraphics[width=1\textwidth]{figuras/ferramentasOrquestracao.jpg}
 \end{figure}
 
A ferramenta escolhida por \citeonline{tcc-gabriel} e a qual ser� utilizada no desenvolvimento deste trabalho � o CloudStack \cite{cloudstack}. A escolha se d� aos motivos de que a ferramenta demonstra uma evolu��o constante num projeto que mant�m uma comunidade participativa, documenta��o completa e detalhada e o uso de tecnologias conhecidas para armazenamento de dados.


\section{Sistemas Multiagentes}
Sistemas multiagentes � considerada uma �rea de estudo dentro da Intelig�ncia Artificial Distribu�da (IAD)\abreviatura{IAD}{Intelig�ncia Artificial Distribu�da}. SMA possuem a capacidade de lidar com problemas em ambientes distribu�dos e abertos, tal como os ambientes em larga escala encontrados que usam a internet como meio \cite{wooldridge2009introduction}.\par

A abordagem de SMA � descrita na forma de m�ltiplos elementos computacionais (agentes) que trocam conhecimento entre si na forma de coopera��o, coordena��o, negocia��o e similares, possibilitando a fragmenta��o de problemas complexos em sub-problemas menores e objetivos. Estes, por sua vez, podem ser abordados de diferentes formas, por diferentes agentes especialistas \cite{a-ricardo-intro}.\par

O emprego de SMA tem apresentado sucesso em �reas que trabalham com ambientes din�micos e descentralizados, onde a tomada de decis�o n�o depende apenas de um �nico ponto de vista \cite{kelash2007takes}.


\subsection{Agentes}
Agentes s�o entidades capazes de se comunicar entre si, possuem mobilidade e comportam-se de forma independente e inteligente dentro do ambiente. Um agente � um sistema computacional capaz de atuar autonomamente de maneira a alcan�ar seus objetivos \cite{wooldridge2009introduction}. Seu conceito pode ser  abstraido para uma entidade auto-contida, capaz de interagir (atrav�s de fun��es atuadoras e sensoras) com o 
ambiente em que se encontra, consigo mesma e com outros agentes \cite{ia-ferramentas}. Ter v�rios agentes em um sistema implica que cada agente deve raciocinar e agir levando em considera��o os outros agentes no sistema, sendo que dentro de um sistema, diferentes agentes podem ou n�o ter um prop�sito em comum \cite{handbook-knowledge}. Este segundo caso � particularmente interessante, j� que as diferentes entidades podem ter o mesmo objetivo, mas diferentes crit�rios, possivelmente conflitantes, a priorizar na satisfa��o de suas metas. Independente de seu tipo e ambiente, agentes funcionam em um ciclo cont�nuo de perceber o mundo, decidir sua pr�xima a��o e execut�-la. A Figura \ref{fig:agente} representa o ciclo cont�nuo de execu��o de um agente. 

\begin{figure}[!htb]
	\centering
	\caption{Representa��o de um agente em um ambiente, obtido de \cite{RussellLivro}.}\label{fig:agente}
	\includegraphics[width=1\textwidth]{figuras/imgAgent.png}
\end{figure}

Agentes possuem m�dulos que servem como sensores e atuadores, representando sua interface com o mundo. Eles est�o diretamente relacionados com as capacidades dos agentes de perceber e agir. Atrav�s dos sensores, um agente consegue perceber o ambiente que ele est� inserido. Esta entrada, na forma percep��es, � avaliada pelo agente, resultando em uma decis�o de a��o. Por fim, atrav�s dos atuadores, o agente � capaz de agir em seu ambiente, modificando o estado do ambiente.

\subsubsection{Caracter�sticas de Agentes}

Um agente � respons�vel por representar um ponto de vista e agir de forma coerente para alcan�ar um objetivo. Assim como uma sociedade de agentes exprime v�rios pontos de vista e objetivos. No evento em que dois ou mais pontos de vista entram em conflito, os agentes devem ter a capacidade de negociar, para que possam encontrar a melhor a��o para o problema em conjunto \cite{wooldridge2009introduction}.\par 

Segundo Silveira em \cite{a-ricardo-intro}, mesmo tendo natureza diversa, os agentes possuem caracter�sticas recorrentes, relacionadas com suas abordagens para resolver problemas e capacidade de comunica��o. Encontrado em maior ou menor grau em uma implementa��o de um agente, estes atributos s�o:

\begin{itemize}  
	\item Reatividade: a habilidade de perceber o ambiente de modo seletivo e manifestar um
	comportamento como resposta a um est�mulo externo;
	\item Autonomia: comportamento dirigido a objetivos, pr�-ativo e auto-iniciado;
	\item Comportamento cooperativo: trabalhar com outros agentes para atingir um objetivo
	comum;
	\item Habilidade de comunica��o ao n�vel de conhecimento: capacidade de comunicar-se com
	pessoas ou outros agentes em uma linguagem de mais alto n�vel que um simples protocolo
	de comunica��o programa a programa;
	\item Capacidade de infer�ncia: capacidade de agir a partir de especifica��es abstratas de
	tarefas, usando conhecimentos pr�vios;
	\item Continuidade temporal: persist�ncia de identidade por longos per�odos de tempo;
	\item Personalidade: capacidade de demonstrar atributos de um personagem;
	\item Adaptabilidade: habilidade de aprender com a experi�ncia;
	\item Mobilidade: habilidade de migrar de uma plataforma para outra.
\end{itemize}

As diferen�as entre agentes � o fator respons�vel pela heterogeniedade do SMA. O fato de que diferentes entidades dividem o mesmo ambiente, faz com que um problema possa ser percebido de diferentes formas e lidado de maneiras diferentes e condizentes com um plano de a��o conjunto. Em adi��o � suas caracter�sticas, os agentes possuem estrat�gias diferentes para a tomada de decis�o, fator que gera diferentes classifica��es para estes, descritas nas pr�ximas sub-se��es.

\subsubsection{Agentes Reativos}

Segundo \citeonline{wooldridge2009introduction}, agentes puramente reativos s�o entidades computacionais que decidem suas a��es baseados exclusivamente no estado presente do ambiente. Eles recebem seu nome em raz�o do comportamento an�logo a uma resposta imediata, sem consulta a um hist�rico passado de conhecimento. Essa abordagem � focada no que os agentes podem realizar, juntos ou separadamente, sem existir necessariamente mem�ria ou comunica��o direta entre eles. Nesse caso, agentes tomam conhecimento de a��es e comportamentos de outros 
agentes apenas atrav�s de modifica��es no ambiente \cite{ia-ferramentas}. Embora sejam �teis em raz�o de que s�o programados de forma a ter comportamentos hierarquicamente organizados, esse tipo de agente pode se tornar muito complexo para entender quando o n�mero de comportamentos associado a eles cresce.

\subsubsection{Agentes Deliberativos}

Agentes deliberativos s�o aqueles que tem alguma forma de reten��o de conhecimento na forma de experi�ncias passadas, criando um modelo expl�cito de mundo. Estes agentes tamb�m possuem uma expl�cita representa��o de outros agentes, mem�ria (que 
permite que os agentes plenejem suas a��es tendo eventos passados como base) e capacidade de se comunicarem diretamente uns com os outros. Esse modelo de agente � mais focado nas atitudes que os caracterizam, sendo que com o tempo, realizando diferentes intera��es com os sistemas, estes devem desenvolver cren�as (no sentido de reconhecer padr�es), inten��es, entre outras caracter�sticas de uma entidade racional com a finalidade de auxiliar a escolha de seus planos de a��o futuros, suplementando sua percep��o atual do ambiente \cite{handbook-knowledge}.\par

No contexto de um agente cognitivo, um conjunto de cren�as � chamado de banco de conhecimento. Esse � continuamente atualizado com as percep��es do agente, assim como � usado para verificar condi��es necess�rias para a escolha de um pr�ximo plano de a��o do agente. Na Figura \ref{fig:agente-BDI} pode-se observar a arquitetura de um agente deliberativo que age no formato de \textit{Beliefs, Desire and Intentions} \abreviatura{BDI}{Beliefs Desire Intentions}(BDI), em portugu�s cren�as, desejos e inten��es.

\begin{figure}[!htb]
	\centering
	\caption{Modelo gen�rico de um agente deliberativo BDI, obtido de \cite{bdi4jade}.}\label{fig:agente-BDI}
	\includegraphics[width=1\textwidth]{figuras/bdiArch.jpg}
\end{figure}

Os agentes BDI s�o aqueles que mant�m uma s�rie de objetivos, armazenados na forma de desejos, mant�m planos na forma de inten��es e possuem cren�as, que s�o informa��es que guardam de forma simb�lica sobre o ambiente. Utilizando l�gica de predicados de primeira ordem, um agente BDI pode escolher qual plano melhor se encaixa na sua vis�o de mundo atual.

\subsubsection{Agentes H�bridos}

De acordo com \cite{wooldridge2009introduction}, agentes h�bridos s�o capazes de apresentar comportamento reativo e proativo atrav�s da modelagem de camadas de comportamentos. A partir dos dados providos pela leitura do ambiente, o agente pondera seus comportamentos da camada de planos reativos e da sua camada de comportamentos proativos. A sa�da dessa fun��o de pondera��o � interpretada de forma an�loga a uma sugest�o de plano. Agentes deste tipo funcionam gra�as a presen�a expl�cita ou n�o de uma camada de controle de comportamentos.

\subsection{Modelagem de Sistemas de Agentes}
Sistemas baseados em agentes podem ser modelados utilizando-se t�cnicas de engenharia de software que, assim como a abordagem orientada a objetos, se constitui em uma outra abordagem denominada modelagem orientada a agentes, que aborda um n�vel mais alto de abstra��o, buscando identificar as nuances envolvidas com os pap�is dos agentes, seus objetivos individuais e coletivos e suas intera��es m�tuas. V�rias abordagens de Engenharia de Software Orientada a Agentes \abreviatura{AOSE}{\textit{Agent Oriented Software Engineering}}(AOSE na sigla em ingl�s) s�o encontradas na literatura. Algumas inspirados na orienta��o a objetos e outras bastante diversa \cite{intelligent}. Em consequ�ncia disso, m�todos relacionados ao desenvolvimento de aplicativos orientados a objeto formam a base de boa parte dos m�todos de desenvolvimento de sistemas baseados em agentes. De forma similar a t�cnica para modelagem orientada a objeto proposta por \citeonline{OMT} que consiste de modelo b�sico, modelo est�tico e modelo din�mico, \citeonline{agent-oriented} divide a an�lise orientada a agentes em tr�s modelos, o modelo de agente, o modelo organizacional e o modelo cooperativo, descritos a seguir:

\begin{itemize}
  \item O modelo de agente cont�m descri��es e estruturas internas dos agentes, descritas em termos de no��es mentais como metas planos e cren�as ou quaisquer estruturas que sejam apropriadas a arquitetura espec�fica dos agentes sendo desenvolvidos. Esse modelo se assemelha ao modelo b�sico de m�todos de orienta��o a objeto;
  \item O modelo organizacional especifica os relacionamentos entre agentes e seus tipos. Estes s�o em parte rela��es de heran�a, e tamb�m relacionamentos entre agentes baseados em seus respectivos pap�is em organiza��es. Essas organiza��es podem ser meios para estruturar sistemas complexos em subsistemas (assim como � feito em certas t�cnicas de orienta��o a objeto) ou podem ser usadas para modelar organiza��es reais. Esse modelo � semelhante ao modelo est�tico, mas como pap�is podem mudar com o passar do tempo, ele n�o � um modelo genuinamente est�tico;
  \item O modelo cooperativo descreve a intera��o, ou mais especificamente, a coopera��o entre os agentes. Este modelo cont�m apenas os \hyphenation{relacio-namentos} relacionamentos entre objetos. O processo que ocorre dentro dos objetos � representado pelo modelo de agente.
\end{itemize}


\subsection{Metodologias para Modelagem de Agentes}
A partir dos conceitos de modelagem de agentes apresentados, foram elaboradas diversas metodologias para a concep��o e o desenvolvimento destes agentes. Em fun��o da sua utilidade em rela��o ao desenvolvimento de agentes \emph{BDI}, sem deixar de ser �til para qualquer outro tipo de sistema multiagente, bem como a abund�ncia de documenta��o e ferramentas de modelagem dispon�veis, a metodologia utilizada no decorrer deste trabalho � a chamada \emph{Prometheus} \cite{prometheus}.

A metodologia \emph{Prometheus} consiste de tr�s fases, descritas por \citeonline{prometheus} da seguinte forma:
\begin{itemize}
  \item Especifica��o de Sistema: � focada em identificar as funcionalidades b�sicas do sistema como um todo, tais como suas entradas (percep��es), sa�das (a��es), e quaisquer fontes importantes de dados compartilhados;
  \item Modelagem Arquitetural: utiliza as sa�das da fase anterior para determinar quais agentes o sistema conter� e como eles ir�o interagir;
  \item Modelagem Detalhada: � onde s�o especificados os componentes internos de cada agente e como tal agente realizar� suas tarefas dentro do sistema.
\end{itemize}

A figura 5 demonstra estas tr�s fases citadas acima de forma expl�cita, mostrando os artefatos gerados por cada fase, bem como os eventos que levam de uma fase a outra.

\begin{figure}[!htb]
	\centering
	\caption{Diagrama explicitando as fases da metodologia \emph{Prometheus}, obtido de \cite{prometheus}.}\label{fig:promehteus}
	\includegraphics[width=1\textwidth]{figuras/prometheus.png}
\end{figure}

Vale notar que estas fases descritas acima e explicitadas na figura 5 constituem um processo iterativo de engenharia de software, e n�o um modelo de desenvolvimento do tipo cascata, sendo que estas fases n�o devem ser necessariamente executadas em alguma ordem em particular. \citeonline{prometheus} sugerem repetir o processo inteiro mais de uma vez, com um foco diferente a cada repeti��o, sendo que a primeira itera��o pode consistir inteiramente de atividades associadas a fase de especifica��o de sistema, itera��es subsequentes envolver�o uma mistura de atividades de fases diferentes, eventualmente com uma presen�a maior de atividades das fases seguintes.


\subsection{Ferramentas para Modelagem de Agentes}
Dentre as ferramentas para modelagem de agentes dispon�veis, a \emph{Prometheus Design Tool} \cite{pdt} chama aten��o por ter sido elaborada em conjunto com os autores da metodologia \emph{Prometheus}, justamente para complementar a metodologia.

A \emph{Prometheus Design Tool} � baseada em caracter�sticas comumente associadas a agentes como metas, planos, percep��es, a��es e protocolos e � estruturada em torno das tr�s fases da metodologia \emph{Prometheus} descritas anteriormente \cite{pdt}. Al�m do suporte ao desenvolvimento gr�fico de diagramas de \emph{design} para o sistema multiagente, \citeonline{pdt} citam diferentes atributos da ferramenta, sendo alguns dos mais not�veis destes:

\begin{itemize}
  \item Verifica��o de Consist�ncia: a ferramenta mant�m restri��es de nomea��o baseadas em um modelo de meta-dados, providenciando suporte � preven��o de erros simples como a gera��o de uma entidade indesejada causada por um erro de digita��o;
  \item Propaga��o: sempre que poss�vel, informa��es s�o propagadas de uma parte do modelo para outra. Por exemplo, se um conjunto de metas � associado a um papel, e esse papel � associado a um agente, ent�o as metas ser�o automaticamente associadas ao agente;
  \item Testes Automatizados: a ferramenta suporta a gera��o e execu��o automatizadas de testes de unidade para eventos, cren�as e planos, baseados no modelo. H� tamb�m a possibilidade de executar testes criados pelo usu�rio.
\end{itemize}



\subsection{Comunica��o de Agentes}
A coopera��o e a comunica��o entre agentes torna-se interessante quando existe a inten��o de resolver um problema de forma distribu�da, como � o caso da orquestra��o de um ambiente de computa��o em nuvem. Existem v�rios m�todos de comunica��o entre agentes fundamentalmente diferentes, sendo que \citeonline{intelligent} classificam como o mais simples dos m�todos a invoca��o do procedimento de um agente por outro agente.

Utilizando a invoca��o de procedimentos, o agente invocador (agente 1) usa par�metros de invoca��o expl�citos para informar o outro agente (agente 2) de suas inten��es. Os valores de retorno do agente 2 representam a resposta da comunica��o. Por�m, apenas comunica��es muito simples podem ser efetuadas invocando procedimentos remotos, sendo que para casos com um m�nimo de complexidade os m�todos de comunica��o entre agentes podem ser diferenciados em sistemas \emph{blackboard} (quadro negro) e sistemas baseados em di�logo (troca de mensagens) \cite{intelligent}.

O conceito de quadro negro surgiu a partir da inten��o de lidar com aplica��es complexas e fracamente definidas, dando mais flexibilidade a pesquisadores e desenvolvedores, liberando-os de especifica��es formais excecivas \cite{blackboard}. Um quadro negro proporciona a todos os agentes dentro de um sistema uma �rea comum de trabalho, a qual eles podem utilizar para trocar informa��es \cite{intelligent}. Um agente inicia a comunica��o escrevendo um item de informa��o qualquer no quadro negro. Este item fica ent�o dispon�vel para todos os outros agentes do sistema, sendo que qualquer agente pode acessar o quadro negro a qualquer momento para verificar se novas informa��es surgiram desde sua �ltima verifica��o \cite{intelligent}. Neste modelo os agentes n�o precisam saber dos conhecimentos e nem da exist�ncia dos outros agentes dentro do sistema, mas devem ser capazes de compreender as informa��es contidas no quadro negro \cite{blackboard}.

A alternativa ao quadro negro vem na forma de troca de mensagens. A comunica��o via troca de mensagens proporciona uma base flex�vel para a implementa��o de estrat�gias complexas de coordena��o de agentes, sendo que as mensagens trocadas entre os agentes podem ser usadas para estabelecer comunica��es e mecanismos de coopera��o usando protocolos pr�-definidos \cite{intelligent}. O fato de que diferentes agentes podem ser invocados para realizarem fun��es diferentes dentro do ambiente, mas em algum ponto de suas respectivas fun��es, necessitar de alguma informa��o da qual outro agente pode ter conhecimento \cite{handbook-intelligence} � algo que torna essa abordagem interessante em um ambiente de computa��o em nuvem.


\subsection{Padr�es de Comunica��o}
Com o �mbito de iniciar a padroniza��o das linguagens de sistemas aut�nomos, a ag�ncia dos estados unidos, \textit{Defense Advanced Research Projects Agency} \abreviatura{DARPA}{Defense Advanced Research Projects Agency}(DARPA) \cite{darpasite}, iniciou um movimento chamado de \abreviatura{KSE}{Knowledge Sharing Effort}\textit{Knowledge Sharing Effort} (KSE) \cite{ksesite} que mais tarde geraria o primeiro padr�o de mensagens para comunica��o de agentes, o \textit{Knowledge Query and Manipulation Language} \abreviatura{KQML}{Knowledge Query and Manipulation Language}(KQML) \cite{finin1994kqml}.\par

O KQML � uma linguagem para comunica��o de agentes, servindo como um envelope sem�ntico para o conte�do da mensagem. O padr�o define um formato comum, para as trocas de informa��o entre agentes, sem interferir no conte�do em si. Segundo Wooldridge em \cite{wooldridge2009introduction}, o envelope � composto de par�metros na forma de chave-valor. Estes tem a fun��o de garantir caracter�sticas tais como independ�ncia de linguagem e suporte a diferentes representa��o interna de conhecimento. Dessa forma o receptor pode, entre outras coisas, saber a linguagem em que o remetente escreveu e espera a resposta e ter acesso a ontologia do remetente. Os par�metros da mensagens s�o complementados por um elemento performativo, que � an�logo ao tipo da mensagem (ex. pergunta, resposta, oferta, entre outros). Seguindo um racic�nio similar a orienta��o a objetos, pode-se pensar em performativas como a classe de um objeto mensagem e seus par�metros ( pares chave-valor ) como vari�veis de uma inst�ncia.\par

Atrav�s da padroniza��o de mensagens, agentes podem interpretar de forma diferente o conte�do de uma mensagem, fazendo com que diferentes cabe�alhos gerem significados diferentes para um mesmo conte�do.\par 

Segundo \cite{wooldridge2009introduction}, o KQML foi criticado, entre outras raz�es, pela aus�ncia de performativas necess�rias para atos de coordena��o entre agentes e por ter sem�ntica pouco rigorosa, levando a mensagens amb�guas. Outros padr�es foram ent�o propostos e introduzidos no cen�rio de comunica��o de agentes.\par

\subsubsection{Especifica��es FIPA}

A FIPA \abreviatura{FIPA}{Foundation for Intelligent Physical Agents}(\emph{Foundation for Intelligent Physical Agents}) � uma organiza��o que promove tecnologia baseada em agentes e a interoperabilidade de seus padr�es com outras tecnologias \cite{fipa}. As v�rias especifica��es promovidas pela FIPA abrangem diferentes categorias, sendo essas:
\begin{itemize}
	\item Comunica��o de Agentes;
	\item Transporte de Agentes;
	\item Gerenciamento de Agentes;
	\item Arquitetura abstrata de agentes;
	\item Aplicativos.
\end{itemize}
Dentre as categorias listadas, \cite{fipa} define comunica��o de agentes como sendo a parte principal de seu modelo de sistema multiagente como um todo.\par

O primeiro documento produzido pela FIPA, chamado FIPA97 \cite{fipa97}, especifica as regras normativas que permitem que uma sociedade de agentes interopere, ou seja, existir, operar e serem gerenciados \cite{jade-fipa}. As regras foram desenvolvidas englobando os preceitos de reusabilidade e interoperabilidade entre SMAs. O padr�o FIPA usa a mesma estrutura proposta pelo KQML, por�m suas cole��es de a��es performativas divergem. A linguagem de comunica��o entre agentes FIPA tem como ponto forte uma sem�ntica baseada em teorias de atos de di�logos mapeados para a��es racionais. Dessa forma, as mensagens podem representar cren�as, vontades e incertezas de um agente de forma n�o amb�gua \cite{serugendo2007autonomous}.\par

Al�m disso, \citeonline{fipa97} tamb�m especifica a \abreviatura{ACL}{Agent Communcation Language}ACL (\emph{Agent Communcation Language}), uma linguagem voltada especificamente para a comunica��o entre agentes. A ACL especifica a codifica��o, sem�ntica e pragm�tica das mensagens trocadas entre agentes, mas n�o especifica um mecanismo ou t�cnologia a ser usada para a troca de mensagens, fazendo com que implementa��es usando \textit{Java multi-threads}, \abreviatura{CORBA}{Common Object Request Broker Architecture}CORBA (\emph{Common Object Request Broker Architecture}), entre outros possam acordar com o padr�o FIPA. Em fun��o da diversidade de t�cnologias que as plataformas podem empregrar para realizar o transporte de mensagem e de que agentes distintos podem residir em plataformas de terceiros, possivelmente utilizando tecnologias de rede diferentes, a FIPA especifica que mensagens que s�o transportadas entre plataformas desconexas devem ser codificadas em forma de texto \cite{jade-fipa}.\par

� importante ressaltar que o padr�o FIPA, segundo o documento de especifica��es FIPA97, descreve tamb�m o modelo de refer�ncia para plataformas de sistemas multiagentes, sendo identificados agentes e fun��es chaves como termos necess�rios para o controle da plataforma. Os agentes indispens�veis para manuten��o e funcionamento da plataforma, mostrados na Figura \ref{fig:fipa}, s�o:

\begin{figure}[!htb]
	\centering
	\caption{Refer�ncia de plataformas de agente do padr�o FIPA, retirada de \cite{jade-fipa}.}\label{fig:fipa}
	\includegraphics[width=1\textwidth]{figuras/fipa.png}
\end{figure}


\begin{itemize}
	\item \textit{Agent Management System}\abreviatura{AMS}{Agent Management System} (AMS) � o agente que deve supervisionar o acesso e uso da plataforma. Ele � respons�vel por autenticar os agentes internos e controlar seus registros;
	
	\item \textit{Agent Communication Channel}\abreviatura{ACC}{Agent Communication Channel} (ACC) � o agente que pr�ve o caminho para contatos b�sicos entre agentes dentro e fora da plataforma. Ele � o m�todo padr�o de comunica��o que oferece um servi�o confi�vel, preciso e ordenado de mensagens. Este agente deve ser respons�vel tamb�m por trazer suporte para interoperabilidade entre diferentes plataformas de agentes;
	
	\item \textit{Directory Facilitator}\abreviatura{DF}{The Directory Facilitator} (DF) � o agente que pr�ve um servi�o de p�ginas amarelas � plataforma de agentes.
\end{itemize}


\subsection{Ferramentas existentes}
Esta se��o � designada a apresentar as plataformas de SMA mais utilizadas, seus pontos fortes, fracos e caracter�sticas.

\subsubsection{JADE}

\textit{Java Agent DEvelopment} \abreviatura{JADE}{\textit{Java Agent Development}}(JADE)\cite{jade} � um \textit{framework} para desenvolvimento de aplica��es de agentes compat�vel com as especifica��es da FIPA para interoperabilidade entre sistemas multiagentes, desenvolvido inicialmente pela Telecom Italia (que ainda � detentora de direitos autorais sobre a ferramenta), � distribu�da em formato de c�digo aberto sob as condi��es da LGPLv2 \cite{lgpl2}. O objetivo da ferramenta JADE � providenciar ao desenvolvedor uma maneira simples de desenvolver seu programa, facilitando a implementa��o dos padr�es da FIPA. As caracter�sticas do JADE importantes para o problema apresentado neste trabalho s�o:

\begin{itemize}
	\item Plataforma que segue o padr�o FIPA, incluindo os tr�s agentes, necess�rios e descritos, ativados junto com a inicializa��o da plataforma;

	\item Plataforma de agentes distribu�da. A plataforma pode ser dividida entre v�rios servidores, contanto que n�o haja \textit{firewall} entre eles, usando apenas uma aplica��o em cada servidor, rodando em uma \textit{Java Virtual Machine}\abreviatura{JVM}{\textit{Java Virtual Machine}} (JVM). Agentes s�o implementados como uma \textit{thread} Java e suas a��es paralelas s�o tratadas como tarefas, estas escalonadas pela pr�pria plataforma de maneira leve e eficiente. A comunica��o entre agentes na mesma JVM se d� pelo uso de \textit{Java Events};

	\item Transporte de mensagens dentro da mesma plataforma � feito atrav�s de transfer�ncias de objetos Java para evitar convers�es de dados. Quando o receptor ou remetente n�o perten�e a mesma plataforma, a mensagem � automaticamente convertida para o formato FIPA;

	\item Suporte a agentes reativos e a agentes BDI a partir de um modelo de agente gen�rico.

\end{itemize}

As JVMs instanciadas atuam, cada uma, como um container de agentes do JADE. Cada container � respons�vel por gerenciar localmente o ciclo de vida de seus agentes instanciados, assim como deve manter as informa��es necess�rias para realizar a comunica��o dos seus agentes com agentes residentes em outros containers, tais como um \textit{buffer} contendo o endere�o de agentes externos cujo foi trocado mensagens recentemente.\par

Uma JVM do JADE ir� assumir o papel de fachada da plataforma para o ambiente externo. Este container especial possui os agentes de ger�ncia descritos pela FIPA e � respons�vel por representar toda a plataforma para com plataformas de SMA de terceiros. Assim, esta inst�ncia � respons�vel, dentre outras coisas, por fazer o roteamento das mensagens trocadas entre agentes internos e externos � plataforma, registro de agentes e descoberta de agentes em outras plataformas. A Figura \ref{fig:jadearq} representa a arquitetura do JADE de forma visual.

\begin{figure}[!htb]
	\centering
	\caption{Arquitetura da plataforma JADE, obtida de \cite{jade-site}.}\label{fig:jadearq}
	\includegraphics[width=1\textwidth]{figuras/jadearq.png}
\end{figure}

Agentes devem ser capazes de se comunicar com outros agentes tanto dentro da mesma plataforma quanto com agentes externos a esta. Os agentes externos s�o aqueles alocados em outros SMAs fora do dom�nio da aplica��o. Para que a troca de mensagens seja poss�vel, ambas as plataformas devem respeitar o mesmo protocolo.\par

Toda a comunica��o de agentes feita pelo JADE acontece na forma de troca de mensagens no formato FIPA. Entretanto as tecnologias usadas para alcan�ar agentes diferem de acordo com a localiza��o do destinat�rio em rela��o ao remetente. A comunica��o intra-plataforma (entre agentes residentes do mesmo container ou de containers diferentes) � implementada usando solu��es fornecidas pela t�cnologia Java. Por�m a comunica��o onde o destino � externo � plataforma s�o utilizados protocolos \textit{Internet Inter-ORB Protocol}\abreviatura{IIOP}{\textit{Internet Inter-ORB Protocol}} (IIOP), sendo este um protocolo para sistemas distribu�dos que possibilita a comunica��o entre sistemas escritos em linguagens diferentes \cite{jade}. Os diferentes cen�rios de troca de mensagens s�o descritos da seguinte forma:

\begin{itemize}
	\item Mesma plataforma, mesma JVM -- as mensagens s�o trocadas sem invoca��o remota, atrav�s da chamada do m�todo \textit{clone()} em um objeto Java \textit{ACLMessage};

	\item Mesma Plataforma, JVM diferente -- as mensagens s�o trocadas atrav�s de \textit{Remote Method Invocation}\abreviatura{RMI}{\textit{Remote Method Invocation}} (RMI), uma interface IIOP para objetos Java. O objeto \textit{ACLMessage} � serializado no remetente e des-serializado no destinat�rio;

	\item Plataformas diferentes, ambas JADE -- uma chamada direta entre ACCs � feito usando o padr�o CORBA. Desta forma, um objeto Java � transformado em literal, depois em \textit{stream} de bytes IIOP pelo remetente e o processo inverso acontece no destinat�rio;

	\item Plataformas diferentes (n�o-JADE) -- mesmo caso que o anterior, a �nica mudan�a est� no destinat�rio, onde o processo de tratamento do \textit{stream} de bytes no padr�o FIPA � tratado como uma caixa preta.
\end{itemize}

A plataforma JADE usa as facilidades e pacotes do Java para prover as funcionalidades de sistemas multiagentes, sendo uma plataforma de f�cil uso e entendimento para programadores j� habituados com a linguagem, atrav�s do uso de recursos como \textit{Java Annotations} e orienta��o � objetos. Entretanto, o suporte ao desenvolvimento de agentes em outras linguagens de programa��o dentro da plataforma n�o � um assunto endere�ado no \textit{framework}.


\subsubsection{JADEX}

JADEX � um \textit{framework} de software para a cria��o de agentes que seguem o modelo BDI, usando a linguagem Java \cite{jadex}. O nome JADEX deriva da ferramenta JADE, sendo que o JADEX surgiu como uma expans�o para o JADE, adicionando suporte � arquitetura BDI \cite{jadex}. O JADEX teve sua origem na Universidade de Hamburgo e � distribu�do atualmente em c�digo aberto sob os termos da licen�a GPLv3 \cite{gpl3}. A comunidade em torno do JADEX n�o � t�o ativa quanto a comunidade do JADE. Em compensa��o, a documenta��o dispon�vel � mais detalhada do que a
dispon�vel para o JADE, com descri��es do seu funcionamento e modos de uso. Tutoriais explicando as diversas fun��es da ferramenta e guias para iniciantes s�o abundantes.\par

A motiva��o da ferramenta � prover a funcionalidade de criar agentes cognitivos orientados a BDI para plataformas de SMA com agentes gen�ricos. O \textit{framework} usa de t�cnicas de orienta��o a objetos para realizar a modelagem de conceitos comuns ao modelo de agentes cognitivos, tais como \cite{pokahr2005jadex}:

\begin{itemize}
	\item Cren�as: modeladas tanto no formato de fatos ou conjunto de fatos.  Mudan�as na base de cren�as podem ser feitas de forma descritiva, modificando diretamente o conjunto de cren�as de um agente;

	\item Objetivos: s�o o conceito principal do Jadex. Eles s�o tratados como desejos moment�neos de um agente e modelados de forma a representar um estado alvo do mundo. Esta modelagem � composta de um conjunto de fatos necess�rios para a completude do objetivo. Um agente ir� sempre, direta ou indiretamente, tomar um plano de a��o que visa atingir um conjunto de seus objetivos. Este conceito � modelado de forma a ser acess�vel pelos planos do agente, facilitando a cria��o e modifica��o de objetivos caso haja a necessidade;

	\item Planos: representam elementos comportamentais dos agentes Jadex. S�o compostos por um cabe�alho e um corpo. No cabe�alho de um plano est�o descritas as pr�-condi��es que far�o com que este plano seja adotado. No corpo � detalhado, de forma procedural, como o agente deve agir caso esse plano seja escolhido para execu��o;

	\item Capacidades: s�o mecanismos para agrupamento de elementos BDI. Este conceito tem como objetivo trazer reusabilidade e organiza��o para elementos relacionados. Capacidades s�o definidas como um conjunto de cren�as, planos, objetivos e eventos. Este conceito � associado aos agentes, resultando em agentes que podem ter acesso a elementos e atributos de suas capacidades.

\end{itemize}

O Jadex faz uso de \abreviatura{XML}{\textit{Extensible Markup Language}} \textit{extensible markup language} (XML) para implementar o \textit{agent definition file}\abreviatura{ADF}{\textit{Agent Definition File}} (ADF), um arquivo que descreve os elementos BDI, tais como o agente. A implementa��o destes elementos, no entanto, � feita atrav�s da programa��o de objetos Java, implementando interfaces dispon�veis pelo Jadex. A Figura \ref{fig:jadexagent} apresenta um agente Jadex resumido, explicitando suas partes descritiva e procedural.

\begin{figure}[!htb]
	\centering
	\caption{Representa��o de um agente JADEX \cite{pokahr2005jadex}.}\label{fig:jadexagent}
	\includegraphics[scale=0.45]{figuras/jadex2.png}
\end{figure}

O JADEX � executado sobre a plataforma do JADE. A integra��o ocorre atrav�s de uma ferramenta de integra��o desenvolvida para injetar os agentes JADEX dentro do banco de agentes do JADE. Estes agentes s�o ent�o instanciados atrav�s de modelos criados a partir de seus arquivos descritores e seus comportamentos s�o mapeados como \emph{cyclic behaviours}, uma forma de comportamento dispon�vel provido pela plataforma JADE. Dessa forma, os planos s�o executados passo a passo e mant�m acesso sobre todas as caracter�sticas do agente. O JADEX, dessa forma, acaba servindo como uma extens�o para a plataforma JADE, complementando-a com a possibilidade de especificar agentes cognitivos complexos que usam a estrat�gia de BDI para tomada de decis�es.



\bibliographystyle{ufscThesis/ufsc-alf}
\bibliography{bibliografia}

\end{document}
